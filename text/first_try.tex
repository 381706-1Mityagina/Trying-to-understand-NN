\documentclass{article}
\documentclass[utf8x, 12pt]{G7-32} % Стиль (по умолчанию будет 14pt)
\usepackage{longtable}
\usepackage{graphicx}
\usepackage{mathtext}
\usepackage[numberright]{eskdplain}
\usepackage[utf8]{inputenc}
\usepackage[T1]{fontenc}
\usepackage{amsmath,amsthm,amssymb}
\usepackage{mathtext}
\usepackage[T1,T2A]{fontenc}
\usepackage[english,bulgarian,ukranian,russian]{babel}
\usepackage[argument]{graphicx}
\graphicspath{{pics/}}
\DeclareGraphicsExtensions{.pdf,.png,.jpg}

\usepackage{geometry} % Меняем поля страницы
\geometry{left=2cm}% левое поле
\geometry{right=1.5cm}% правое поле
\geometry{top=1cm}% верхнее поле
\geometry{bottom=2cm}% нижнее поле

\begin{document}
    \begin{center} 
    \large МИНИСТЕРСТВО ОБРАЗОВАНИЯ И НАУКИ РОССИЙСКОЙ ФЕДЕРАЦИИ

федеральное государственное автономное образовательное учреждение высшего образования

\textbf{Национальный исследовательский нижегородский государственный университет им. Н.И. Лобачевского}

Институт информационных технологий, математики и механики \\
\textbf{Кафедра математического обеспечения и суперкомпьютерных технологий}\\[3.5cm] 
    
    \huge \textbf{Отчёт по практике} \\[0.6cm] % название работы, затем отступ 0,6см
    \\ 
    \huge{На тему :}\\[0.6cm]
    \huge \textbf{Разработка и оптимизация программ под MyriadX VPU}\\[5.7cm]
    
    
    \end{center} 
    
    \begin{flushright}
    \large \textbf{Выполнил:} 
    студент гр. 381706-1 \\
    Митягина Дарья Сергеевна \\ [0.65cm]
    --------------------------------------\\
                                Подпись\\
    \textbf{Научный руководитель:} \\
    Профессор, доктор технических наук\\
    Турлапов Вадим Евгеньевич \\ [0.65cm]
    --------------------------------------\\
                                Подпись\\
    [3.2cm]
    \end{flushright}
    
    % \vfill 
    
    \begin{center} 
    \large Нижний Новгород 2020
    \end{center} 

    % \thispagestyle{empty}
    \newpage
      \begin{center}
    %   Содержание
      \end{center}
        \tableofcontents
      
    \newpage
    \begin{center} 
    \section{Введение}
    %   \huge \textbf{Введение} \\[1.3cm]
    \end{center}
    \large 
    
    Эффективное использование ускорителей глубокого обучения является важной частью продуктивного использования искусственного интеллекта (далее ИИ).
    
    Чтобы преодолеть проблемы с ограниченной производительностью в одноядерных однопроцессорных системах, стали популярными параллельные компьютерные системы с несколькими процессорами (или несколькими процессорными ядрами).
    
    Однако в работе с подобными устройствами есть и некоторые труности. Программирование параллельных приложений часто представляет собой более сложную задачу, чем программирование последовательных приложений. А значит, параллельные компьютерные системы часто требуют, чтобы программист заботился о деталях низкого уровня, которые могут различаться между разными системы.\\ \\
    Целью данной работы является :\\
    \begin{enumerate} 
    \item обзор существующих аппаратных решений в области deep learning'а
    \item детальный обзор архитектуры Myriad X VPU
    \item описание принципов программирования под эту платформу
    \item результатов некоторых проведенных экспериментов
    \end{enumerate} 
    \\
    
    \newpage
    \begin{center} 
    \section{Выбор аппаратного обеспечения}
    \end{center}
    
    Глубокое обучение требует больших вычислительных ресурсов, поэтому очень важно, какое аппаратное обеспечение будет выберано для исследований. Рассмотрим возможные варианты.
    
    Существует множество видов процессоров, но не все подходят для быстрой и эффективной работы с нейросетями.\\
    \begin{enumerate} 
    \item 
    Графический процессор (graphics processing unit, GPU) — отдельное устройство персонального компьютера или игровой приставки, выполняющее графический рендеринг; в начале 2000-х годов графические процессоры стали массово применяться и в других устройствах: планшетные компьютеры, встраиваемые системы, цифровые телевизоры.
    
    Современные графические процессоры очень эффективно обрабатывают и отображают компьютерную графику, благодаря специализированной конвейерной архитектуре они намного эффективнее в обработке графической информации, чем типичный центральный процессор.
    
    Высокая вычислительная мощность GPU объясняется особенностями архитектуры. Современные CPU содержат небольшое количество ядер, тогда как графический процессор изначально создавался как многопоточная структура с множеством ядер. Разница в архитектуре обусловливает и разницу в принципах работы. 
    \item 
    Тензорный процессор (tensor processing unit, TPU) — тензорный процессор, относящийся к классу нейронных процессоров, являющийся специализированной интегральной схемой.
    
    По сравнению с графическими процессорами, рассчитан на более высокий объём вычислений с пониженной точностью (например, всего 8-разрядную точность).
    
    Устройство реализовано как матричный умножитель для 8-разрядных чисел, управляемый CISC-инструкциями центрального процессора по шине PCIe 3.0. Изготавливается по технологии 28 нм, тактовая частота составляет 700 МГц и имеет тепловую расчётную мощность 28—40 Вт. Оснащается 28 Мбайт встроенной оперативной памяти и 4 Мбайт 32-разрядных аккумуляторов, накапливающих результаты в массивах из 8-битных множителей, организованных в матрицу размером 256×256. Инструкции устройства передают данные на узел или получают их из него, выполняют матричные умножения или свёртки. В такт может производиться 65536 умножений на каждой матрице; в секунду — до 92 трлн.
    \item
    Программируемая пользователем вентильная матрица(field-programmable gate array, FPGA) — полупроводниковое устройство, которое может быть сконфигурировано производителем или разработчиком после изготовления; наиболее сложная по организации разновидность программируемых логических интегральных схем.
    
    Программируются путём изменения логики работы принципиальной схемы, например, с помощью исходного кода на языке описания аппаратуры. Могут быть модифицированы практически в любой момент в процессе их использования. Cостоят из конфигурируемых логических блоков, подобных переключателям с множеством входов и одним выходом (логические вентили, gates). Принципиальное отличие ППВМ состоит в том, что и функции блоков, и конфигурация соединений между ними могут меняться с помощью специальных сигналов, посылаемых схеме. В некоторых специализированных интегральных схемах (ASIC) используются логические матрицы, аналогичные ППВМ по строению, однако они конфигурируются один раз в процессе производства, в то время как ППВМ могут постоянно перепрограммироваться и менять топологию соединений в процессе использования.
    \item
    Процессор машинного зрения (vision processing unit, VPU) — новый класс специализированных микропроцессоров являющихся разновидностью ИИ-ускорителей, предназначенных для аппаратного ускорения работы алгоритмов машинного зрения.
    
    VPU во многом похожи на тензорные процессоры, но они узкоспециализированы для ускорения работы алгоритмов машинного зрения, в которых используются методы свёрточных нейронных сетей (CNN) и масштабно-инвариантная трансформация признаков (SIFT). В них делается большой акцент на распараллеливание потока данных между множеством исполнительных ядер, включая использование модели блокнотной памяти — как в многоядерных цифровых сигнальных процессорах, и они так же, как тензорные процессоры, используются для вычислений c низкой точностью, принятой при обработке изображений.\\
    Именно этот вид будет рассмотрен подробнее.
    \end{enumerate} 
    
    \newpage
    \begin{center}
    \section{Архитектура MyriadX VPU}
    \end{center}
    
    \begin{center} 
    \includegraphics[scale=0.6]{arch4.png}
    \\
    \caption{Рис. 1 - }
    \\ \\
    \end{center} 
    \\
    % \begin{center} 
    % \includegraphics[scale=1.5]{arch.jpg}
    % \\ \\
    % \caption{Рис. 2 - }
    % \\
    % \end{center} 
    
    \newpage
    продолжение
    \newpage
    продолжение
    \newpage
    продолжение
    \newpage
    продолжение
    
    \newpage
    \begin{center} 
    \section{Принципы программирования под MyriadX VPU}
    \end{center}
    
    \newpage
    продолжение
    \newpage
    продолжение
    
    \newpage
    \begin{center} 
    \section{Краткий обзор источников}
    \end{center}
    
    \newpage
    продолжение
    \newpage
    продолжение
    \newpage
    продолжение
    \newpage
    продолжение
    
    \newpage
    \begin{center} 
    \section{Выполненные задачи}
    \end{center}
    
    \newpage
    продолжение
    \newpage
    продолжение
    \newpage
    продолжение
    \newpage
    продолжение
    \newpage
    продолжение
    
    \newpage
    \begin{center} 
    \section{Заключение}
    \end{center}
    
    \newpage
    продолжение
    
    \newpage
    \begin{center} 
    \section{Список литературы}
    \end{center}
    
    \newpage
    \begin{center} 
    \section{Приложение}
    \end{center}
    
    \newpage
    продолжение
    \newpage
    продолжение
    
    \newpage
    \begin{center} 
    \section{Теоретический минимум}
    \end{center}
    
    \newpage
    продолжение
    
    \end{document}
