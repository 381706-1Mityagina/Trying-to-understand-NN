\documentclass{article}
\documentclass[utf8x, 12pt]{G7-32} % Стиль (по умолчанию будет 14pt)
\usepackage{longtable}
\usepackage{graphicx}
\usepackage{mathtext}
\usepackage[numberright]{eskdplain}
\usepackage[utf8]{inputenc}
\usepackage[T1]{fontenc}
\usepackage{amsmath,amsthm,amssymb}
\usepackage{mathtext}
\usepackage[T1,T2A]{fontenc}
\usepackage[english,bulgarian,ukranian,russian]{babel}
\usepackage[argument]{graphicx}
\graphicspath{{pics/}}
\DeclareGraphicsExtensions{.pdf,.png,.jpg}

\usepackage{geometry} % Меняем поля страницы
\geometry{left=2cm}% левое поле
\geometry{right=1.5cm}% правое поле
\geometry{top=1cm}% верхнее поле
\geometry{bottom=2cm}% нижнее поле

\begin{document}
    \begin{center} 
    \large МИНИСТЕРСТВО ОБРАЗОВАНИЯ И НАУКИ РОССИЙСКОЙ ФЕДЕРАЦИИ

федеральное государственное автономное образовательное учреждение высшего образования

Национальный исследовательский нижегородский государственный университет им. Н.И. Лобачевского

Институт информационных технологий, математики и механики \\
Кафедра математического обеспечения и суперкомпьютерных технологий\\[3.5cm] 
    
    \huge Отчёт по практике \\[0.6cm] % название работы, затем отступ 0,6см
    \\ 
    \huge{Тема :}\\[0.6cm]
    \huge Сети Generative adversarial networks для синтеза изображений по текстовому описанию\\[7.7cm]
    
    
    \end{center} 
    
    \begin{flushright}
    \large \textbf{Выполнил:} \\
    студент гр. 381706-1 \\
    Митягина Дарья Сергеевна \\
    \textbf{Научный руководитель:} \\
    Профессор, доктор технических наук\\
    Турлапов Вадим Евгеньевич \\
    [3.7cm]
    \end{flushright}
    
    % \vfill 
    
    \begin{center} 
    \large Нижний Новгород 2020
    \end{center} 
    
    \thispagestyle{empty}
    \newpage
      \begin{center}
    %   Содержание
        \tableofcontents
      \end{center}
    \newpage
    \begin{center} 
    \section{Введение}
    %   \huge \textbf{Введение} \\[1.3cm]
    \end{center} 
    
    \newpage
    \begin{center} 
    \section{Краткий обзор статей}
    %   \huge \textbf{Статья ""} \\[1.3cm]
    \end{center}
    \large 
    В данной работе будут рассмотрены четыре статьи, две из которых будут разобраны более подробно.
    \\ \\
    \textit{1. Статья GAN синтез изображения из текста
    \\
    "Generative Adversarial Text to Image Synthesis", Scott Reed, Zeynep Akata, Xinchen Yan, Lajanugen Logeswaran, Bernt Schiele, Honglak Lee}
    \\

    В этой работе авторы задались целью перевести текст (человеческие письменные описания) непосредственно в пиксели изображения.
    \\
    Подход, предложенный авторами, заключается в обучении DC-GAN (Deep Convolutional Generative Adversarial Network), обусловленом текстовыми признаками, закодированными гибридной сверточно-рекуррентной нейронной сетью (CRNN) на уровне символов. Обе сети G (генератор) и сеть D (дискриминатор) выполняют вывод с прямой связью, обусловленный текстовым признаком.
    \\ \\
    \textit{2. Статья Text2Scene: создание композиционных сцен из текстовых описаний
    \\
    "Text2Scene: Generating Compositional Scenes from Textual Descriptions", Fuwen Tan, Song Feng, Vicente Ordonez}
    \\

    В этой работе авторы представляют Text2Scene, модель для интерпретации визуально описательного языка для генерации композиции. Они специально сосредоточелись на создании представления сцены, состоящего из списка объектов вместе с их атрибутами (например, местоположение, размер, пропорции, поза, внешний вид). Описанный метод, в отличие от многих, не опирается на порождающие состязательные сети (GAN). Вместо этого создается интерпретируемая модель, которая итеративно генерирует сцену, предсказывая и добавляя новые объекты на каждом временном шаге.
    
    \newpage
    \begin{center} 
    \section{Статья ""}
    %   \huge \textbf{Статья ""} \\[1.3cm]
    \end{center} 
    
    \newpage
    \begin{center} 
    \section{Статья ""}
    %   \huge \textbf{Статья ""} \\[1.3cm]
    \end{center} 
    
    \newpage
    \begin{center} 
    \section{Практика}
    %   \huge \textbf{Практика} \\[1.3cm]
    \end{center} 
    
    \newpage
    \begin{center} 
    \section{Заключение}
    %   \huge \textbf{Заключение} \\[1.3cm]
    \end{center} 
    
    \newpage
    \begin{center} 
    \section{Список литературы}
    %   \huge \textbf{Список литературы} \\[1.3cm]
    \end{center} 
    
\end{document}
