\documentclass{article}
\documentclass[utf8x, 12pt]{G7-32} % Стиль (по умолчанию будет 14pt)
\usepackage{longtable}
\usepackage{graphicx}
\usepackage{mathtext}
\usepackage[numberright]{eskdplain}
\usepackage[utf8]{inputenc}
\usepackage[T1]{fontenc}
\usepackage{amsmath,amsthm,amssymb}
\usepackage{mathtext}
\usepackage[T1,T2A]{fontenc}
\usepackage[english,bulgarian,ukranian,russian]{babel}
\usepackage[argument]{graphicx}
\graphicspath{{pics/}}
\DeclareGraphicsExtensions{.pdf,.png,.jpg}

\usepackage{geometry} % Меняем поля страницы
\geometry{left=2cm}% левое поле
\geometry{right=1.5cm}% правое поле
\geometry{top=1cm}% верхнее поле
\geometry{bottom=2cm}% нижнее поле

\begin{document}
    \begin{center} 
    \large МИНИСТЕРСТВО ОБРАЗОВАНИЯ И НАУКИ РОССИЙСКОЙ ФЕДЕРАЦИИ

Федеральное государственное автономное образовательное учреждение

Высшего образования

«Нижегородский государственный университет им. Н.И. Лобачевского»

Национальный исследовательский университет

Институт информационных технологий, математики и механики 
Кафедра математического обеспечения и суперкомпьютерных технологий\\[3.5cm] 
    
    \huge Отчёт по практике \\[0.6cm] % название работы, затем отступ 0,6см
    \\ 
    \huge{Тема :}\\[0.6cm]
    \large Generative adversarial networks for Text-to-Image Synthesis\\[7.7cm]
    
    
    \end{center} 
    
    \begin{flushright}
    \textbf{Выполнил:} \\
    студент гр. 381706-1 \\
    Митягина Д.С. \\
    \textbf{Научный руководитель:} \\
    Профессор, доктор технических наук\\
    Турлапов В.Е. \\
    [4.7cm]
    \end{flushright}
    
    
    % \vfill 
    
    \begin{center} 
    \large Нижний Новгород 2019
    \end{center} 
    
    \thispagestyle{empty}
    \newpage
      \begin{center}
    %   Содержание
      \end{center}
    \newpage
    \begin{center} 
      \huge \textbf{Введение} \\[1.3cm]
    \end{center} 
      \large Oдной из самых сложных проблем в мире Computer Vision является синтез высококачественных изображений из текстовых описаний. Без сомнения, это интересно и полезно, но современные системы искусственного интеллекта далеки от этой цели.

    Генерация изображений имеет множество возможных применений в
    будущем, когда технологии будут готовы для коммерческого применения. 
    Люди смогут создавать схему расположения мебели для своего дома, просто описывая ее на компьютере, а не тратя много часов на поиск нужного дизайна.\\ 
    Создатели контента смогут творить в более тесном сотрудничестве с машиной, используя естественный язык.
    Кроме того, данная тема может стать более интересной, если ее развить. А именно, перевести задачу из генерации 2d изображения в построение 3d сцены и даже, возможно, формирование видео материалов с использование полученных сцен.\\ \\
    В данном отчете будут \\
    - рассмотрены уже существующие решения задачи генерации изображений по текстовому описанию\\
    - разобраны основные составляющие части программной реализации этих решений\\
    - описаны основные положения из теории, лежащие в основе выбранных методов\\
    - приведен их сравнительный анализ\\
    - предъявлены результаты некоторых проведенных экспериментов\\
    \newpage
    \begin{center} 
    \huge \textbf{Постановка задачи} \\[1.3cm]
    \end{center} 
      \large 1. Исследовать различные публикации на поставленную тему.\\
             2. Изучить средства и методы решения задачи синтеза изображений из их словесного описания.\\
             3. Начать проведение практических экспериментов.\\
    \newpage
    \begin{center} 
    \huge \textbf{Необходимый теоретический минимум} \\[1.3cm]
    \end{center} 
      \large \textbf{1. Граф сцены}\\
      Граф сцены представляет структуру, которая содержит логическое и зачастую (но не обязательно) пространственное представление графической сцены. Определение графа сцены нечёткое, поскольку программисты, осуществляющие его реализацию в приложениях, — и, в частности, в индустрии разработки игр — берут базовые принципы и адаптируют их для применения в конкретных приложениях. Это означает, что нет договорённости о том, каким должен быть граф сцены.

    Граф сцены представляет собой набор узлов такой структуры, как граф или дерево. Узел дерева (в предельной структуре дерева графа сцены) может иметь множество потомков, но зачастую только одного предка, причём действие предка распространяется на все его дочерние узлы; эффект действия, выполненного над группой, автоматически распространяется на все её элементы. Во многих программах ассоциирование матрицы преобразования (см. также трансформации и матрицы) на уровне любой группы и умножение таких матриц представляет собой эффективный и естественный способ обработки таких действий. Общей особенностью, к примеру, является способность группировать связанные формы/объекты в составной объект, который можно перемещать, трансформировать, выбирать и т. д. так же просто, как и одиночный объект. \\ \\
    \textbf{2. Свёрточная нейронная сеть}\\
    Идея свёрточных нейронных сетей заключается в чередовании свёрточных слоёв (англ. convolution layers) и субдискретизирующих слоёв. Структура сети — однонаправленная (без обратных связей), принципиально многослойная. Для обучения используются стандартные методы, чаще всего метод обратного распространения ошибки. Функция активации нейронов (передаточная функция) — любая, по выбору исследователя.  \\ \\
    \textbf{3. Генеративно-состязательная сеть (GAN)}\\
    Алгоритм машинного обучения без учителя, построенный на комбинации из двух нейронных сетей, одна из которых (сеть G) генерирует образцы,а другая (сеть D) старается отличить правильные («подлинные») образцы от неправильных. Так как сети G и D имеют противоположные цели — создать образцы и отбраковать образцы — между ними возникает Антагонистическая игра. Генеративно-состязательную сеть описал Ян Гудфеллоу из компании Google в 2014 году.

    Использование этой техники позволяет в частности генерировать фотографии, которые человеческим глазом воспринимаются как натуральные изображения.  \\ \\
    \textbf{4. Архитектура нейронной сети}\\
    1. Входные узлы (входной слой): вычислений в этих слоях нет, они просто передают информацию следующему слою.\\
    2. Скрытые узлы (скрытый слой): в скрытых слоях выполняется промежуточная обработка или вычисления, после чего происходит перенос весов с входного слоя на следующий слой.\\
    3. Выходные узлы (выходной слой): здесь мы наконец используем функцию активации.\\
    4. Соединения и веса: сеть состоит из соединений, каждое соединение передает выход нейрона i на вход нейрона j. В этом смысле i является предшественником j, а j является преемником i. Каждому соединению присваивается вес $W_{i,\;j}$.\\
    5. Функция активации: функция активации узла определяет выходные данные этого узла с учетом входных данных или набора входных данных. \\
    6. Правило обучения. Правило обучения - это правило или алгоритм, который изменяет параметры нейронной сети для того, чтобы данный вход в сеть создавал предпочтительный результат. Этот процесс обучения обычно сводится к изменению весов и порогов.\\
    
    \newpage
    \begin{center} 
    \huge \textbf{Разбор статей} \\[0.5cm]
      \huge Image Generation from Scene Graphs.\\
      Justin Johnson, Agrim Gupta, Li Fei-Fei\\ [1.3cm]
    \end{center} 
      \large Большинство существующих методов дают потрясающие результаты на ограниченных доменах, таких
как описания птиц или цветов, но не могут точно воспроизвести
сложные предложения со многими объектами и отношениями.\\
Для преодоления этого ограничения в данной статье авторы предлагают метод
генерирования изображений из графов сцен, позволяющих явно рассуждать об
объектах и их отношениях. Представленная в публикации модель использует свертку графа для
обработки входныч графов, вычисляет макет сцены, прогнозируя границы
Bounding Box (BBox, ограничивающий параллелепипед) и маски сегментации для объектов, и преобразует макет в
изображение с каскадным уточнением сети.\\ \\
Основы метода :\\
Предложение представляет собой линейную структуру, в которой одно слово следует за другим; однако,
информация, передаваемая сложным предложением, часто может быть более явно представлена в виде графа сцены
объектов и их отношений.\\ \\
 - Для обработки входных данных графа сцены используется сеть свертки графа.\\
 - Для создания изображения, которое соответствует макету применяется cascaded refinement network (CRN), которая
обрабатывает макет.\\
 - Чтобы убедиться, что сгенерированные изображения реалистичны и содержат необходимые объекты разумно применить
генеративно-состязательные сети, работающие на патчах изображений и сгенерированных объектах.\\
 - Сквозной процесс обучения можно разделить на два основных компонента :\\ 
 Учебный компонент - первый этап, на котором машина записывает все параметры, выполняемые оператором (через
 Сверточные нейронные сети (CNN)). \\
  Компонент логического вывода возможен тогда, когда машина действует на основе ранее полученного опыта от компонента обучения сквозного процесса обучения. \\
  \includegraphics[scale=0.5]{scheme.jpg}
  \\
  \begin{center} 
  \caption*{Рис. 1 - Схема, описывающая метод}
  \end{center} 
  \\
  \begin{center} 
  \includegraphics[scale=0.5]{sheep.jpg}
  \\
  \caption*{Рис. 2 - Генерация сцена графа по предложению}
  \end{center} 
\end{document}
