\documentclass{article}
\documentclass[utf8x, 12pt]{G7-32} % Стиль (по умолчанию будет 14pt)
\usepackage{longtable}
\usepackage{graphicx}
\usepackage{mathtext}
\usepackage[numberright]{eskdplain}
\usepackage[utf8]{inputenc}
\usepackage[T1]{fontenc}
\usepackage{amsmath,amsthm,amssymb}
\usepackage{mathtext}
\usepackage[T1,T2A]{fontenc}
\usepackage[english,bulgarian,ukranian,russian]{babel}

\usepackage{geometry} % Меняем поля страницы
\geometry{left=2cm}% левое поле
\geometry{right=1.5cm}% правое поле
\geometry{top=1cm}% верхнее поле
\geometry{bottom=2cm}% нижнее поле

\begin{document}
    \begin{center} 
    \large МИНИСТЕРСТВО ОБРАЗОВАНИЯ И НАУКИ РОССИЙСКОЙ ФЕДЕРАЦИИ

Федеральное государственное автономное образовательное учреждение

Высшего образования

«Нижегородский государственный университет им. Н.И. Лобачевского»

Национальный исследовательский университет

Институт информационных технологий, математики и механики \\[3.5cm] 
    
    \huge Отчёт по практике \\[0.6cm] % название работы, затем отступ 0,6см
    \large Generative adversarial networks for Text-to-Image Synthesis\\[8.7cm]
    
    
    \end{center} 
    
    \begin{flushright}
    Выполнил: студент гр. 381706-1 \\
    Митягина Дарья \\[4.7cm]
    \end{flushright}
    
    
    % \vfill 
    
    \begin{center} 
    \large Нижний Новгород 2019
    \end{center} 
    
    \thispagestyle{empty}
\end{document}