\documentclass{article}
\documentclass[utf8x, 12pt]{G7-32} % Стиль (по умолчанию будет 14pt)
\usepackage{longtable}
\usepackage{graphicx}
\usepackage{mathtext}
\usepackage[numberright]{eskdplain}
\usepackage[utf8]{inputenc}
\usepackage[T1]{fontenc}
\usepackage{amsmath,amsthm,amssymb}
\usepackage{mathtext}
\usepackage[T1,T2A]{fontenc}
\usepackage[english,bulgarian,ukranian,russian]{babel}
\usepackage[argument]{graphicx}
\graphicspath{{pics/}}
\DeclareGraphicsExtensions{.pdf,.png,.jpg}

\usepackage{geometry} % Меняем поля страницы
\geometry{left=2cm}% левое поле
\geometry{right=1.5cm}% правое поле
\geometry{top=1cm}% верхнее поле
\geometry{bottom=2cm}% нижнее поле

\begin{document}
    \begin{center} 
    \large МИНИСТЕРСТВО ОБРАЗОВАНИЯ И НАУКИ РОССИЙСКОЙ ФЕДЕРАЦИИ

федеральное государственное автономное образовательное учреждение высшего образования

Национальный исследовательский нижегородский государственный университет им. Н.И. Лобачевского

Институт информационных технологий, математики и механики \\
Кафедра теории управления и динамики систем \\[3.5cm] 
    
    \huge Отчёт по практике \\[0.6cm] % название работы, затем отступ 0,6см
    \\ 
    \huge{Тема :}\\[0.6cm]
    \huge Lower Frequencies\\[7.7cm]
    
    
    \end{center} 
    
    \begin{flushright}
    \large \textbf{Выполнил:} \\
    студент гр. 381703-2 \\
    Диженин Владислав Евгеньевич \\
    \textbf{Научный руководитель:} \\
    Специалист, инженер\\
    Середа Яна Александровна \\
    [3.7cm]
    \end{flushright}
    
    % \vfill 
    
    \begin{center} 
    \large Нижний Новгород 2020
    \end{center} 
    
    \thispagestyle{empty}
    \newpage
      \begin{center}
    %   Содержание
      \end{center}
        \tableofcontents
      
    \newpage
    \begin{center} 
    \section{Введение}
    %   \huge \textbf{Введение} \\[1.3cm]
    \end{center}
    \large 
    При обучении сетей есть низкие частоты и высокие. И чел вывел теорему, точнее доказал. Что чем ниже частота, тем быстрее будет происходить обучение. \\ ЗАМЕНИИИИИ\\
    Цели данной работы:\\
    \begin{enumerate} 
    \item изучение теории в данной области
    \item обзор реализованных программных решений поставленной задачи
    \item исследование существующих публикаций на выбранную тему
    \item изложение результатов некоторых проведенных экспериментов
    \end{enumerate} 
    \\
    
    \newpage
    \begin{center} 
    \section{Краткий обзор источников}
    \large \textbf {Статья On the Spectral Bias of Neural Networks, \\ Nasim Rahaman, Aristide Baratin, Devansh Arpit, Felix Draxler, Min Lin. Fred A. Hamprecht, Yoshua Bengio,Aaron Courville}    
    \end{center}
    \\
    \large 
    В этой работе авторы задались целью показать, что более низкая частота компоненты обученных сетей более устойчива к случайным возмущениям параметров. 
    \\
    Авторам статьи удалось:
    \begin{enumerate}
    \item применить непрерывную кусочно-линейную структуру сетей ReLU для оценки спектра Фурье
    \item найти эмпирическое доказательство спектрального смещения: т.е. более низкие частоты изучаются первыми. 
    \item изучить роль формы многообразия данных: показать, как сложные формы многообразия могут облегчить изучение высоких частот и разработку теоретического понимания этого поведения
    \end{enumerate}
    \\ \\
    Хотя нейронные сети могут аппроксимировать произвольные функции, авторы статьи находят, что эти сети приоритезируют изучение низкочастотных режимов, явление, называемое спектральным смещеним. Этот уклон проявляется не просто в процессе обучения, но и в параметризации модели: на самом деле, показано, что более низкая частота компоненты обученных сетей более устойчива к случайным возмущениям параметров.
    \\
    
    \\
    \newpage
    \begin{center} 
    \section{Результаты практики}
    %   \huge \textbf{Практика} \\[1.3cm]
    \end{center}
    \large
    
    \newpage
    \begin{center} 
    \section{Заключение}
    %   \huge \textbf{Заключение} \\[1.3cm]
    \end{center}
    \large
    В ходе работы над данным проектом были достигнуты следующие цели:\\
    \begin{enumerate}
        \item изучены доступные решения поставленной задачи
        \item получены некоторые базовые знания
        \item часть этих знаний применена на практике
    \end{enumerate}
    \\ \\ \\
    В процессе изучения данной области, конечно, были и неудачи. Не удалось реализовать эффективный метод решения поставленной задачи, готова лишь часть необходимого функционала.
    \\ \\
    Можно сделать вывод о том, что данная область полна потенциала и является актуальной для исследования.
    
    \newpage
    \begin{center} 
    \section{Приложение. Термины и определения.}
    %   \huge \textbf{Введение} \\[1.3cm]
    \end{center} 
    \large 
    \textit{1.}
    \textbf{Анализ Фурье} — направление в анализе, изучающее каким образом общие математические функции могут быть представлены либо приближены через сумму более простых тригонометрических функций. 
    \\ \\
    \textit{2.}
    \textbf{ReLU} - В последние годы большую популярность приобрела функция активации под названием «выпрямитель» (rectifier, по аналогии с однополупериодным выпрямителем в электротехнике). Нейроны с данной функцией активации называются ReLU (rectified linear unit). ReLU имеет следующую формулу f(x) = max(0, x) и реализует простой пороговый переход в нуле.
    \\ \\
    \textit{3.  []}
    \\
    \\ \\
    \textit{4.  []}
    \\
    \\ \\
    \textit{5.  []}
    \\
    \\ \\
    \textit{6. []}
    \\
    \\ \\
    \textit{7. []}
    \\
    \\ \\
    \textit{8.  []}
    \\
    \\ \\
    \textit{9. []}
    \\
    \\ \\
    
    \newpage
    \begin{center} 
    \section{Список литературы}
    %   \huge \textbf{Список литературы} \\[1.3cm]
    \end{center}
    \large
    \begin{enumerate} 
    \item 
    \item 
    \item 
    \item 
    \item 
    \item 
    \item 
    \item 
    \item 
    \item 
    \item 
    \item 
    \item 
    \end{enumerate}
    
\end{document}
